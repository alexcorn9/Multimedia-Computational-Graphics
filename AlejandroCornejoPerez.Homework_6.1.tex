\documentclass[12pt]{article}
\usepackage[top=2cm]{geometry}

\title{Homework 6.1: VideoLAN Client Biography}
\author{Alejandro Cornejo Pérez}
\date{February 2026}
\begin{document}

\maketitle
The VideoLAN Client, now most commonly known as VLC Media Player, is a free and open source multimedia project developed by the VideoLAN project. It was created in 1996 by students from the École Centrale Paris in France. The software began as a college project so that the university could have a network where it streamed videos. It was designed so well that it was rewritten 2 years later and released to the general public which allowed general collaboration and development across the globe.

VLC evolved from a client server streaming system to a standalone media player after they integrated server functionality into the main program, in the present, it is developed by a worldwide community of volunteers that are coordinated by the non-profit VideoLAN organization. One of the most important and crucial aspects of VLC is the ability to be used across many platforms, it can be used on Operative Systems like Windows, macOS, Linux, Android, iOS, etc. This makes it one of the most accessible media players worldwide.

VLC has had a lot of important contributions for the Media Players industry. Firstly, it has universal media playback which means that it can play nearly all video and audio formats in the world without requiring additional Codecs, this is because it has built-in libraries, compression and protocols. Another important contribution is that it is one of the few Media Players that is a free and open use software which allows the public all over the world to improve, customize, and innovate on it. Furthermore, VLC has very good Streaming and DVD playback capabilities, it can stream media over networks, convert file formats, encode video, and support encrypted DVD playback. Finally, other important contributions are that it has Global accessibility and Ongoing technological innovation. It has billions of downloads, availability in over 100 languages and recent developments have included AI-powered real-time subtitle generation and translation.

In conclusion, VLC media player has a very interesting origin because it started as a university networking experiment and then became one of the most used media players in the world. VLC along with Linux, represent the power of open-source collaboration. Its ability to play almost any media format, operate across platforms, support streaming and conversion, and continuously innovate is and inspiration for any project.

\end{document}
